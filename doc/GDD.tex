\documentclass[12pt]{article}

\usepackage{sbc-template}

\usepackage{graphicx,url}

%\usepackage[brazil]{babel}   
%\usepackage[latin1]{inputenc}  
\usepackage[utf8]{inputenc}  
% UTF-8 encoding is recommended by ShareLaTex

\sloppy

\title{Bananamen}

\author{Diego P. da Jornada, Eduardo P. Lima Hoefel, Gabriel M. Carlos}


\address{Faculdade de Informática -- Pontifícia Universidade Católica do Rio
    Grande do Sul\\  (PUCRS)\\
  \email{\{diego.jornada, eduardo.hoefel, grabriel.carlos\}@acad.pucrs.br}
}

\begin{document} 

\maketitle

\section{Introdução}

	Esta é uma versão do clássico Bomberman, um jogo de ação
	e puzzle, cujo objetivo é completar as fases colocando
	bombas de forma estratégica para destruir obstáculos e
	inimigos. O jogo difere do original por utilizar bananas
	como bombas. Isto é explicado pela história do jogo, a
	seguir.
	
	``No ano de 2534, depois de não existir mais nada de útil
	para ser descoberto ou inventado, cientistas resolvem
	levar o termo `banana de dinamite' ao pé da letra. A
	fruta é modificada genéticamente, passando a entrar em
	estado de combustão quando em contato com o ar. Os
	cientistas também adicionaram à fruta a essência do
	Guaraná Antártica\textregistered.  Como é sabido em todo
	o universo, o refrigerante possui a famosa Energia Que
	Contagia\texttrademark. Com a combustão, esta energia é
	liberada, gerando uma súbita explosão.  Todo o processo
	leva 3 segundos para ocorrer, contados a partir do
	momento em que a banana foi separada do cacho.

	Os cientistas tentaram manter a fruta em segredo, porém
	um grupo de marginais invadiu o laboratório e roubou o
	produto, levando-o à venda no merdado negro. Você joga
	no papel de um cidadão tentando sobreviver em meio ao
	caos.''

\section{Características}

	O universo de Bomberman possui a clássica arena, onde o
	jogador é colocado junto com inimigos (outros jogadores)
	e precisa eliminá-los para se libertar. Para isso o
	jogador conta com bananas que podem ser plantadas 
	para explodir inimigos ou obstáculos. Porém, o jogador 
	deve tomar cuidado, já que a bomba que ele plantou pode
	explodí-lo também. Existem também NPCs que podem ajudar 
	ou atrapalhar o jogador, entregando itens ou "comendo"
	as suas bombas.

	A jogabilidade envolvida neste jogo é bem simples: o
	jogo fica em uma tela única com uma vista de cima. Nesta
	tela o jogador pode mover-se horizontalmente ou
	verticalmente. Ao colocar uma banana, que levará algum
	tempo até que se exploda, permitindo-lhe tempo para
	fugir.

	Se a bomba atingir um obstáculo, ele será destruído. 
	O jogo é multiplayer com rivalidade, ou seja, os jogadores
	são adversários um do outro. O primeiro jogador pode 
	se mover com as SETAS do teclado e colocar a bomba 
	no RCtrl, o segundo jogador move-se pelas teclas WASD 
	e coloca bomba na tecla ESPAÇO. O jogador que morrer 
	primeiro perde, caso os dois jogadores morram ao mesmo 
	tempo, será considerado empate.

	Os itens podem ser upgrades ou downgrades, e o efeito
	é escolhido aleatoriamente quando o jogador o pega.
	Cada item pode alterar uma das seguintes propriedades:
	\begin{itemize}
	\item Velocidade de movimento;
	\item Cooldown da bomba;
	\item Range da bomba;
	\end{itemize}

	Os NPCs do mapa são macacos, que andam entre os arbustos
	do mapa. Quando uma banana é plantada perto de um macaco,
	ele vai tentar ir até elas para comê-la. Se ele conseguir, 
	a bomba é desarmada. Os macacos também podem colocar um
	item surpresa no chão. Os jogadores não podem interagir
	 ou atacar os macacos, apenas tentar evitá-los.

	O jogo contará com 3 (três) mapas, apenas 1 (um)
	disponível na fase um do projeto.  Os mapas se
	encontram na pasta mapas com a nomeclatura:

	\begin{itemize}
		\item Mapa1 (etapa 1)
		\item Mapa2 (etapa 2)
		\item Mapa3 (etapa 3)
	\end{itemize}

	Os mapas possuem características diferentes,
	dificultando a sobrevivência dos jogadores.

	O jogo possui uma tela inicial (Menu) com a opção de
	começar o jogo e escolher a fase. Há uma tela de pausa
	do jogo, caso o jogador queira pausar durante a partida.
	Além dos avisos “Jogador 1 Venceu!” ou “Jogador 2
	Venceu!”, dependendo de quem ganhe a partida. Emcaso de
	empate aparecerá a mensagem “Empate!” na tela. Após
	alguns segundos são redirecionados para o menu, para
	escolher uma nova fase

\section{Requisitos de arte e áudio}

	Há uma música de fundo no menu, assim como durante a
	partida. Efeitos de áudio quando a bomba for colocada,
	um item for coletado e quando uma bomba explodir. No
	final da partida há um som caso o jogador tenha ganho e
	outro caso tenha perdido.

	Os requisitos de áudio se encontram na pasta áudio
	com a seguinte nomeclatura.

	\begin{itemize}
		\item Menu
		\item Fase1
		\item Fase2
		\item Fase3
		\item ColetaDeItem
		\item Ganhou
		\item Perdeu
		\item Pause
		\item ColocouBanana
		\item ExplodiuBanana
	\end{itemize}

	O projeto conta com os seguintes assets:

	\begin{itemize}
		\item \textbf{Sprites}
			\begin{itemize}

				\item Personagem1
				\item Personagem2
				\item Banana
				\item Explosão

			\end{itemize}
			
		\item \textbf{Mapas}
			\begin{itemize}

				\item Mapa1
				\item Mapa2
				\item Mapa3

			\end{itemize}

		\item \textbf{Áudio}
			\begin{itemize}

				\item Menu
				\item Fase1
				\item Fase2
				\item Fase3
				\item ColetaDeItem
				\item Ganhou
				\item Perdeu
				\item Pause
				\item ColocouBanana
				\item ExplodiuBanana

			\end{itemize}

		\item \textbf{Protótipos}
			\begin{itemize}

				\item PrototipoMenu
				\item Prototipoescolhadefase
				\item Protipopausa

			\end{itemize}
	\end{itemize}

\section{Cronograma}

\end{document}
