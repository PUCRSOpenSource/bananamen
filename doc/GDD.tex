\documentclass[12pt]{article}

\usepackage{sbc-template}

\usepackage{graphicx,url}

\usepackage[brazil]{babel}   
%\usepackage[latin1]{inputenc}  
\usepackage[utf8]{inputenc}  
% UTF-8 encoding is recommended by ShareLaTex

\sloppy

\title{Bananamen}

\author{Diego P. da Jornada, Eduardo P. Lima Hoefel, Gabriel M. Carlos}


\address{Faculdade de Informática -- Pontifícia Universidade Católica do Rio
    Grande do Sul\\  (PUCRS)\\
  \email{\{diego.jornada, eduardo.hoefel, grabriel.carlos\}@acad.pucrs.br}
}

\begin{document} 

\maketitle

\section{Introdução}

	Esta é uma versão do clássico Bomberman, um jogo de ação
	e puzzle, cujo objetivo é completar as fases colocando
	bombas de forma estratégica para destruir obstáculos e
	inimigos.

\section{Características}

	O universo de Bomberman possui a clássica arena, onde o
	jogador é colocado junto com inimigos (outros jogadores)
	e precisa eliminá-los para se libertar. Para isso o
	jogador conta com bombas que podem ser ativadas, uma por
	vez, para explodir inimigos ou obstáculos, tomando
	cuidado, pois a bomba pode explodir o próprio jogador. O
	jogador poderá contar com itens auxiliares, cards que
	encontrará nos obstáculos destruídos, que irão aumentar
	seu poder de ataque e consequentemente suas chances de
	sucesso.

	A jogabilidade envolvida neste jogo é bem simples. O
	jogo fica em uma tela única com uma vista de cima. Nesta
	tela o jogador pode mover-se horizontalmente ou
	verticalmente. Ao colocar uma bomba, que levará algum
	tempo até que se exploda, permitindo-lhe tempo para
	fugir.

	Se a bomba atingir um tijolo ele será destruído, com a
	chance de que no lugar do tijolo apareça um card de
	power up. O jogador que passar por cima desse card terá
	o raio de ataque de sua bomba aumentado, melhorando suas
	chances de eliminar o adversário.  O jogo é multiplayer,
	modo batalha, ou seja, um jogador contra o outro. O
	primeiro jogador pode se mover com as SETAS do teclado e
	colocar a bomba no RCtrl, o segundo jogador move-se
	pelas teclas W,A,S,D e coloca bomba na tecla ESPAÇO. O
	jogador que morrer primeiro perde, caso os dois
	jogadores morram ao mesmo tempo, será considerado
	empate.

	O mapa padrão é um gramado com bordas de concreto que
	impossibilitam o jogador de deixar o cenário. Além de
	blocos de concreto posicionados pela fase. Esses blocos,
	assim como as paredes, não podem ser destruídos. Há
	também tijolos espalhados pelo mapa, que devem ser
	destruídos para abrir caminho, podendo conter itens de
	aprimoramento do personagem (cards de power up).

	O jogo contará com 3 (três) mapas, apenas 1 (um)
	disponível na fase um do projeto.  Os mapas se
	encontram na pasta mapas com a nomeclatura:

	\begin{itemize}
		\item Mapa1 (etapa 1)
		\item Mapa2 (etapa 2)
		\item Mapa3 (etapa 3)
	\end{itemize}

	Os mapas possuem características diferentes,
	dificultando a sobrevivência dos jogadores.

	O jogo possui uma tela inicial (Menu) com a opção de
	começar o jogo e escolher a fase. Há uma tela de pausa
	do jogo, caso o jogador queira pausar durante a partida.
	Além dos avisos “Jogador 1 Venceu!” ou “Jogador 2
	Venceu!”, dependendo de quem ganhe a partida. Emcaso de
	empate aparecerá a mensagem “Empate!” na tela. Após
	alguns segundos são redirecionados para o menu, para
	escolher uma nova fase

\section{Requisitos de arte e áudio}

	Há uma música de fundo no menu, assim como durante a
	partida. Efeitos de áudio quando a bomba for colocada,
	um item for coletado e quando uma bomba explodir. No
	final da partida há um som caso o jogador tenha ganho e
	outro caso tenha perdido.

	Os requisitos de áudio se encontram na pasta áudio
	com a seguinte nomeclatura.

	\begin{itemize}
		\item Menu
		\item Fase1
		\item Fase2
		\item Fase3
		\item ColetaDeItem
		\item Ganhou
		\item Perdeu
		\item Pause
		\item ColocouBomba
		\item ExplodiuBomba
	\end{itemize}

	O projeto conta com os seguintes assets:

	\begin{itemize}
		\item \textbf{Sprites}
			\begin{itemize}

				\item Personagem1
				\item Personagem2
				\item Bomba
				\item Explosão

			\end{itemize}
			
		\item \textbf{Mapas}
			\begin{itemize}

				\item Mapa1
				\item Mapa2
				\item Mapa3

			\end{itemize}

		\item \textbf{Áudio}
			\begin{itemize}

				\item Menu
				\item Fase1
				\item Fase2
				\item Fase3
				\item ColetaDeItem
				\item Ganhou
				\item Perdeu
				\item Pause
				\item ColocouBomba
				\item ExplodiuBomba

			\end{itemize}

		\item \textbf{Protótipos}
			\begin{itemize}

				\item PrototipoMenu
				\item Prototipoescolhadefase
				\item Protipopausa

			\end{itemize}
	\end{itemize}

\section{Cronograma}

\end{document}
